\chapter{ Что объясняет и чего не объясняет антропный принцип?} \label{chapt5}

Рассмотрим, прежде всего, различие между слабым и сильным АП. Оно состоит в следующем. Слабый АП применяется к параметрам, которые зависят от современного возраста Вселенной. Сильный АП применяется к параметрам, которые от возраста не зависят. Иными словами, Слабый АП – предполагается, что законы природы «таковы, каковы они есть, и больше никаковы», а разумная жизнь возникает там, где есть для нее условия. Например, на данный момент считается, что наша Вселенная родилась 14 млрд. лет назад и будет существовать вечно. Почему же мы живем в данную эпоху, относительно близко к моменту ее рождения? Просто потому, что звезд с нужным химическим составом десяток миллиардов лет назад не было и спустя несколько десятков миллиардов лет не будет. Разумная жизнь нашего типа станет невозможной. Сильный АП предполагает, что сами законы природы и параметры типа гравитационной постоянной, массы электрона и т.д. таковы, что должна возникать разумная жизнь.


Обе формулировки, по существу, сводятся к одному и тому же, утверждая, что условия во Вселенной, где есть наблюдатель, должны допускать его существование. 


Таким образом, с помощью АП мы можем объяснить  --- почему во Вселенной наблюдаются те или иные свойства, но не можем объяснить, почему в ней реализовались условия, сделавшие ее обитаемой. Поэтому, полного ответа на вопрос "почему Вселенная такова..." антропный принцип не дает.


%============================================================================================================================

\clearpage
