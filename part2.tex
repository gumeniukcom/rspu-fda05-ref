\chapter{Условия существования жизни} \label{chapt2}

В проблемах, связанных с жизнью во Вселенной, приходится сталкиваться с тремя типами условий: допустимые, необходимые и достаточные. 


Рассмотрим некоторую систему $\mathcal{S}$, в которой реализуются определенные условия. Они могут быть качественные и количественные. Качественные условия означают наличие (или отсутствие) в системе какого-либо свойства (например, наличие атмосферы на планете). Количественные --- выражаются численными значениями некоторых параметров, при этом значения параметров задаются в некотором интервале. Так, можно говорить об определенном интервале температурных условий и т.д. По существу, качественные условия тоже сводятся к количественным. 


Определим теперь необходимые и достаточные условия. Условие будем считать необходимым для жизни в системе $\mathcal{S}$, если при наличии этого условия жизнь в системе $\mathcal{S}$ существует, а при его отсутствии жизнь в ней становится невозможной. Обычно существует целый комплекс необходимых условий. Если, по крайней мере, одно из них не выполняется, жизнь в данной системе невозможна. Поэтому, если выполнена часть из полного набора необходимых условий, то этого недостаточно для существования жизни в рассматриваемой системе. Жизнь в системе $\mathcal{S}$ может существовать тогда и только тогда, когда в ней реализуется весь набор необходимых условий. Этот набор образует комплекс необходимых и достаточных условий. Каждое условие комплекса необходимо для жизни, но только все вместе они являются достаточными. 


С необходимыми условиями тесно связаны непригодные или запрещающие условия. Будем называть так условия, которые исключают существование жизни в системе $\mathcal{S}$, делают ее непригодной для жизни. 


%============================================================================================================================

\clearpage

\chapter{На пути к антропному принципу} \label{chapt3}

Во Вселенной, вероятно, возможны разные формы жизни, но в дальнейшем, говоря о жизни, будем  подразумевать только водно-углеродную форму, к которой принадлежим мы сами. Это делается по той причине, что при обсуждении антропного принципа имеется в виду именно эта форма жизни.


Применительно к этой форме жизни характерной особенностью Вселенной можно считать то, что жизнь существует в ней лишь локально, в ограниченных (и притом очень небольших) областях. Это отличает Вселенную от таких однородных систем, как например, земная биосфера, где жизнь существует повсеместно.


Многие области Вселенной непригодны для жизни. Но поскольку жизнь во Вселенной существует, то условия в ней должны быть допустимыми, они должны допускать существование жизни, хотя бы в некоторых небольших областях. Но какие из допустимых условий во Вселенной можно считать необходимыми для жизни? Очевидно, к ним можно отнести наличие звезд и планет. Чуть менее очевидно, необходимы ли для жизни галактики. Должны ли звезды, чтобы обеспечить возникновение около них жизни, объединяться в огромные системы? Имеют ли эти системы произвольные параметры, или они должны соответствовать какой-то типичной модели галактик? Еще менее очевидно, насколько необходимы для жизни скопления галактик. Насколько, вообще, "глобальные" свойства Вселенной необходимы для жизни? Иными словами, в какой мере существенные черты Вселенной совпадают с жизненно важными параметрами?


Ответ, который дает на этот вопрос антропный принцип оказался весьма неожиданным. Он позволил связать наиболее характерные черты Вселенной, а позднее и фундаментальные свойства материи с существованием во Вселенной жизни и, в том числе, человека.


Оказывается, что основные черты наблюдаемой нами астрономической Вселенной являются характерными для любой обитаемой космической системы. Другими словами, космическая система может стать обитаемой лишь в том случае, если она включает в себя планеты, обращающиеся вокруг звезд, составляющих звездные системы (с параметрами, соответствующими параметрами типичных галактик). А это значит, что наиболее существенные черты наблюдаемой нами Вселенной, ее "глобальные" свойства оказываются необходимыми для возникновения и развития жизни.


Это обстоятельство позволяет понять, почему окружающий нас мир таков, как он есть, почему наблюдаемая Вселенная обладает отмеченными выше свойствами. Это объясняется тем, что мы наблюдаем заведомо не произвольную область Вселенной, а именно ту, особая структура которой сделала ее пригодной для возникновения и развития жизни. Что же касается других областей Вселенной, то в них, могут реализоваться иные физические условия, радикально отличные от наших условий, что делает их непригодными для жизни во всяком случае в известной нам форме. 


На возможное существование связи между наличием условий, допускающих развитие жизни в окружающей нас области Вселенной, и другими характеристиками этой области влияет красное смещение, как один из факторов, благоприятствующих появлению и развитию жизни. Напротив, смена расширения/сжатия привела бы к таким условиям, которые сделали бы невозможным существование жизни.


Таким образом, уже на данном этапе формирования АП были сформулированы две главные относящиеся к нему идеи: 

\begin{itemize}
\itemОсновные черты наблюдаемой Вселенной связаны с существованием в ней жизни (и человека) --- они являются необходимыми для возникновения и развития жизни;
\itemЭто объясняется тем, что мы наблюдаем не произвольную область Вселенной, а ту, в которой существует познающий эту Вселенную субъект (наблюдатель) и в которой реализовались необходимые для его существования условия. 
 \end{itemize}
 
Эти две идеи также можно сформулировать в виде следующего положения:
"Мы являемся свидетелями процессов определенного типа потому, что процессы другого типа протекают без свидетелей".

%============================================================================================================================

\clearpage


