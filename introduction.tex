\chapter*{Введение}							% Заголовок
\addcontentsline{toc}{chapter}{Введение}	% Добавляем его в оглавление

С давних времен люди стремились постичь причины различных процессов, происходящих в природе. Пытались понять, почему определенное явление происходит именно так, а не иначе. Многие вещи, который сейчас нам кажутся простыми и очевидными, например, что яблоко, упавшее с дерева, падает именно вниз, а не вверх, что если бросить камень в воду, он утонет, а деревянная палка --- нет, еще несколько сотен лет назад занимали умы великих ученых.


Со временем знания об окружающем мире накапливались, мировоззрение менялось и пытливым умам уже становилось недостаточно загадок того мира, который нас непосредственно окружает. Тогда взгляды ученых мужей устремлялись в небо --- днем, голубое и спокойное, а ночью полное далеких звезд, и неизведанных тайн.


Ответ на вопрос «Откуда мы все взялись?» уходил все глубже в прошлое, и все дальше от того места, где мы сейчас обитаем. Сначала людей занимала загадка собственного происхождения, затем всего окружающего мира, живого и неживого, осязаемого и бесплотного. Когда этим вопросам было дано логическое объяснение, появилось их разумное продолжение --- «Откуда взялся наш дом – планета «Земля»»? Так постепенно задача нашего происхождения расширилась сначала до рамок Солнечной системы, потом галактики, и, наконец, Вселенной.


Существует множество гипотез о возникновении Вселенной: какие-то из них имеют большую поддержку, какие-то меньшую, некоторые на первый взгляд совсем уж абсурдны, а другие настолько логичны, что сразу кажутся невозможными, но ни одна из них пока не дает ответов на все вопросы, накопившиеся в головах ученых за прошедшие столетия.


В моем реферате речь пойдет не об этом, а о следующей ступени познания нашего происхождения. Я поставлю перед собой вопросы «Почему мы появились именно на это планете, в этой галактике, и в этой Вселенной, если она не единственна? И если это так, то какие они, другие Вселенные, одинаковые ли?». К сожалению, лишь очень поверхностный ответ дает так называемый Антропный принцип, о нем мы и поговорим.






%\textbf{Актуальность} темы состоит в том, что за последние 70 лет ЭВМ вошли  в повседневную жизнь человечества. Темпы роста компьютеризированности человечества лишь возрастают. Информация стала главным ресурсом; а обработка, передача и хранение информации стало  важнейшей деятельностью человечества.  

%\textbf{Целью} данной работы является рассмотреть основные этапы истории развития информатики, взгляды различных ученых на то, что же скрывается за словом "информатика".

%\textbf{Задачами} данной работы является:
%\begin{itemize}
%\item Изучить историю и предысторию информатики;
%\item Сравнить взгляды ученых разных стран на определение "компьютерной науки";
%\item Рассмотреть перспективные направления развития информатики.
%\end{itemize}

\clearpage