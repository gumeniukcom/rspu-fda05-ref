\chapter{Космологический антропный принцип} \label{chapt4}

Для понимания антропного принципа важно понять одно существенное, обстоятельство: он был выдвинут вне всякой связи с проблемой существования разумной жизни или исследованием места человека во Вселенной. Ученых, занимающихся этим вопросом, интересовали совсем другие проблемы: почему тот или иной космологический параметр имеет вполне определенное значение? Почему мир устроен именно так, а не иначе? Почему Вселенная такова, как мы ее наблюдаем? Подход, использованный при решении этой проблемы, соответствует обычной, принятой физике методологии. Если нас интересуют значения каких-то параметров, необходимо попробовать проварьировать эти значения и посмотрим, как изменятся при этом условия в рассматриваемой системе (в данном случае во Вселенной). Этот естественный и вполне логичный подход неожиданно привел к установлению связи между существованием наблюдателя и наблюдаемыми свойствами Вселенной. Покажем это на примерах:

\begin{itemize}
\itemРазмерность физического пространства $\mathcal{N}$. Это одна из важнейших фундаментальных характеристик нашего Мира. Почему пространство имеет только три измерения? Очевидно, при $\mathcal{N}<3$ человек существовать не сможет. Возможно, что существуют двумерные и одномерные миры. Мы можем мысленно изучать их свойства, но наблюдать эти миры мы не в состоянии. 

\itemСредняя плотность вещества во Вселенной. В космологии существует понятие критической плотности $\rho_c$. Если средняя плотность вещества во Вселенной $\rho<\rho_c$, то Вселенная неограниченно расширяется; при $\rho>\rho_c$ расширение сменяется сжатием; при $\rho=\rho_c$ геометрия мира евклидова. Критическая плотность $\rho_c=10^ {-29}$. Средняя плотность "светящегося" вещества, полученная из наблюдений, меньше $\rho_c$, но по порядку величины близка к ней. Если учесть возможно существующую "скрытую массу" во Вселенной, то средняя плотность $\rho$ должна быть еще ближе к критической; может быть она даже превзойдет ее, но останется вблизи $\rho_c$. Итак, во Вселенной удовлетворяется соотношение $\rho$ ~= $\rho_c$. Такое совпадение удивительно, ибо плотность, вообще говоря, может иметь произвольное значение. Какое можно дать этому объяснение? Средняя плотность связана со скоростью расширения Вселенной. Если $\rho << \rho_c$, Вселенная расширяется слишком быстро, и в ней никак не могут образоваться гравитационно-связанные системы --- галактики и звезды, необходимые для жизни. Если $\rho >> \rho_c$, гравитационно-связанные системы легко возникают, но время жизни такой Вселенной (длительность цикла расширение-сжатие) мало, намного меньше, чем требуется для возникновения жизни. Таким образом, если бы условие $\rho \approx \rho_c$ не выполнялось, то жизнь в такой Вселенной была бы невозможна. Следовательно, средняя плотность вещества во Вселенной тоже является жизненно-важным фактором, а условие $\rho \approx \rho_c$ --- необходимым для существования жизни во Вселенной. Это, опять-таки, не объясняет, почему в нашей Вселенной выполняется именно такое соотношение, но позволяет предсказать его для любой обитаемой Вселенной. 

\itemСовпадение больших чисел. Существует несколько удивительных соотношений между константами, характеризующими Вселенную. Они даже получили название "совпадение больших чисел". Является ли оно абсолютно случайным или его можно предсказать теоретически? Оказывается это возможно, но лишь для обитаемой Вселенной.

\itemСтруктура Вселенной и фундаментальные константы. Природа материального мира, его важнейшие свойства в значительной степени определяются фундаментальными физическими константами. К ним, прежде всего, относятся безразмерные константы четырех физических взаимодействий: гравитационного, слабого электромагнитного и сильного, а также массы основных элементарных частиц: протона, нейтрона и электрона. Другие фундаментальные константы, такие как постоянная Планка, гравитационная постоянная, скорость света и заряд электрона включены в определение безразмерных констант. Их значения получены экспериментальным путем. Но почему они имеют именно такие значения? Почему, например, константа гравитационного взаимодействия столь мала? Что было бы, если увеличить ее значение? Что будет, если увеличить массу электрона? 
 \end{itemize}
 
 
Структура Вселенной крайне чувствительна к численным значениям этих постоянных: она сохраняется только в очень узких пределах их изменения. Достаточно значению всего лишь одной из постоянных выйти за эти пределы, как структура Вселенной претерпевает радикальные изменения: в ней становится невозможным существование одного или нескольких базовых структурных элементов --- атомных ядер, самих атомов, звезд или галактик. Во всех этих случаях во Вселенной не может существовать и жизнь. Таким образом, значения фундаментальных констант определяют условия, необходимые для существования во Вселенной жизни (и наблюдателя). 

В любой обитаемой Вселенной (мыслимой или реально существующей) фундаментальные физические константы не могут иметь иные значения, кроме тех, которые известны нам из опыта. Следовательно, используя антропный принцип, мы можем приближенно предсказать значения этих констант, ничего не зная о результатах их экспериментального определения. 

%============================================================================================================================

\clearpage
