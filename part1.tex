\chapter{Основные определения} \label{chapt1}




Основные определения


\textbf{Антропный принцип} --- одно из базовых утверждений современной космологии, согласно которому имеет место удивительная приспособленность Вселенной к существованию в ней человека. Эта приспособленность выражается в наличии очень тонкой подгонки фундаментальных физических констант, при которой даже малые отклонения от их стандартных значений привели бы к такому изменению свойств Вселенной, при котором возникновение в ней человека было бы принципиально невозможно.
     

\textbf{Космология} --- наука, изучающая Вселенную как единое целое, ее строение и эволюцию. Термин космология образован из греческих kosmos – мир, гармония logos – учение, слово. Теоретическим базисом космологии является физическая теория, а ее экспериментальные методы основаны на использовании астрономических наблюдений и специальных космических аппаратов.

\section{Антропоцентрический и антропный принцип} \label{sect1_1}

При рассмотрении антропного принципа (АП) необходимо с самого начала обозначить различия между ним и антропоцентрическим принципом, идущим от Аристотеля. Сходство в наименовании и различные неудачные формулировки АП привели к тому, что в ряде случаев между антропным и антропоцентрическим принципом ставится, по существу, знак равенства. Между тем, содержание этих принципов различно. Антропоцентрический принцип утверждает центральное или, во всяком случае, уникальное, привилегированное положение человека во Вселенной. АП также устанавливает определенное соотношение между фундаментальными свойствами Вселенной в целом и наличием в ней жизни и человека, точнее --- между существованием наблюдателя и наблюдаемыми характеристиками Вселенной. Однако характер этого отношения совершенно иной --- он не требует и не утверждает исключительности человеческого рода.


%============================================================================================================================

\clearpage
