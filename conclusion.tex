\chapter*{Заключение}						% Заголовок
\addcontentsline{toc}{chapter}{Заключение}	% Добавляем его в оглавление


Рассмотрев основные аспекты АП, его отличие от антропоцентрического принципа можно сказать, что его ценность состоит, прежде всего, в том что его помощью удалось наполнить наше представление о допустимых условиях во Вселенной в целом.


Во-вторых очень важной является установленная с помощью АП связь между глобальными характеристиками Вселенной в целом и фундаментальными параметрами материального мира, с одной стороны, и жизненно-важными параметрами --- с другой. 


В-третьих применение антропного принципа показало, что пределы изменения параметров, определяющие необходимые условия, очень узки: достаточно небольшого изменения параметров --- и жизнь во Вселенной становится невозможной. Более того, при этом обнаружилась поразительная взаимосогласованность фундаментальных констант и астрономических свойств Вселенной, демонстрирующая глубокую целесообразность и гармонию физических законов. Это послужило дополнительным поводом для интерпретации АП в духе антропоцентризма, для отождествления его с антропоцентрическим принципом.


Наконец, можно отметить эстетическую роль АП в том, что он утверждает гармонию космоса и человека. Древняя идея о связи между человеком и миром получает здесь качественно новое осмысление.


\clearpage